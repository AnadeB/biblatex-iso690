\documentclass[a4paper,10pt]{ltxdockit}
\usepackage[czech,english]{babel}
\usepackage[utf8]{inputenc}
%\usepackage{tgpagella}
\usepackage[T1]{fontenc}
\usepackage{microtype}
\usepackage{hanging}


\def\t|#1|{\texttt{#1}}
\def\c#1{%
\hangpara{3em}{1}%
\fullcite{#1}}

\newcommand*{\biber}{\sty{biber}\xspace}
\newcommand*{\biblatex}{\sty{biblatex}\xspace}

\usepackage[
   backend=biber
  ,style=iso-authoryear
  ,autolang=other
  ,pagetotal=true
  ,sortlocale=cs_CZ
  ,bibencoding=UTF8
  ,spacecolon=false
  %,block=ragged
]{biblatex}
\addbibresource{mybib.bib}
\titlepage{%
  title={ISO 690 \biblatex style},
  subtitle={},
  url={https://github.com/michal-h21/biblatex-iso690},
  author={Michal Hoftich},
  email={michal.h21@gmail.com},
  revision={0.2.1},
  date={\today}}

\begin{document}
\printtitlepage
\tableofcontents

\section{Introduction}

\subsection{About}

\biblatex is bibliography and citation tool for \LaTeX. This project provides
support for citations and references according to ISO 690 international standard.
As the norm ISO 690 is a little bit ambiguous in some details regarding formatting of records,
we largely follow requirements of Czech interpretation, as it is required form
in many Czech universities. Of course, style can be used in other languages as well.

\subsection{Requirements}

Basically, \biblatex $\geq$ 3.4 with \biber $\geq$ 2.5 is all you need to use this package.
No special packages different from those required by \biblatex package are used. For complete
list of such packages, please refer to the \biblatex documentation.

\subsection{License}

GNU/GPL 3

\subsection{Acknowledgments}

Thanks to all contributors who have participated on the development of this style,
especially Johaness Bötchner, Moewew, Dávid Lupták and others.

\section{Use}

\subsection{General}

\begin{verbatim}
..
\usepackage[english,czech]{babel}
% or in case of using xelatex, use polyglossia package instead
\usepackage{polyglossia}
\setmainlanguage{czech}
\setotherlanguage{english}
..
\usepackage[
   backend=biber
  ,style=iso-authoryear
  ,sortlocale=cs_CZ
  ,autolang=other
  ,bibencoding=UTF8
]{biblatex}
\addbibresource{filename.fileextension}
..
\printbibliography

\end{verbatim}

According to the ISO 690 norm, some of the elements of the bibliographic
resource should be printed in the main document language (language I~am
currently writing) while the others in the language of a resource. You can
specify the language of a resource into the field \t|langid| on the per-entry
basis in a resource (\t|.bib|) file. In addition, all of the languages
specified in these fields have to be loaded by \t|babel| or \t|polyglossia|
package respectively.\label{gen:multilang}

Note that for correct support of localization functionality, \t|babel| package
should be used. The main document language is the last one entered
in a list of languages passed to this package.

\subsection{Package options}

\subsubsection{Provided by \biblatex by default}

Frequently used package options are

\begin{description}
\item[style] style to be used for bibliographic references and citations.
Three possibilities are available, \t|iso-authoryear| commonly known as
Harvard system, \t|iso-numeric| as a numeric system and \t|iso-authortitle|.

\item[backend] backend program for generating bibliographic entries. \biber
is default backend for \biblatex package, providing a big variety
of features. Another options are \t|bibtex| and \t|bibtex8|, but they both
are far behind the possibilities of \biber. \biber is the recommended backend.

\item[autolang] controls which language environment is used. The most
significant value is \t|other|, which supports printing localization
terms in language of resource or language specific hyphenation. Default
value is \t|none|, which disable this feature.

\item[sortlocale] responsible for sorting the bibliography according to the
entered \t|locale| language identifier. The main document language one should
be set.

\item[bibencoding] specifies the encoding of the \t|bib| files. \t|<encoding>|
needs to be explicitly specified only if the encoding of the \t|bib| file
is different from the one of the \t|tex| file. Default value is \t|auto|, i.e.
the encoding of the \t|bib| file is identical to the encoding of the \t|tex|
file.
\end{description}

\subsubsection{Provided by \t|biblatex-iso690| in addition}

\begin{description}
  \item[spacecolon] if \t|true|, the space is printed before the colon
  used in subtitles and publication information. Printing the colon this way
  is not recommended. Default value is \t|false|.

  \item[pagetotal] number of total pages is no longer required if the item
  is being cited as a whole. Setting this option to \t|true| will print
  such optional information in the notes section at the end of the reference.
  Default value is \t|false|.\label{pkg:opt:iso690:pp}

  \item[shortnumeration] the standard ISO 690 allows omission of term
  \t|volume| and terms for smaller components of a serial publication.
  If this option is \t|true|, such terms are distinguish typographically
  (the volume number in bold type and the part number, if required, in
  parentheses). If \t|false|, such terms are printed with preceding
  literal terms.
\end{description}


\subsection{Database guide}

\t|biblatex| supports more entry fields than legacy \t|bibtex|. Hence
some examples of bibliography entries with respective fields follow.

%\begin{description}

\paragraph{Book} \hfill\\

\c{t01}
\begin{verbatim}
@BOOK{t01,
  author      =   {Borgman, Christine L.}, 
  year        =   {2003},
  title       =   {From Gutenberg to the global information infrastructure}, 
  subtitle    =   {access to information in the networked world},
  edition     =   {1},  % should be not stated
  location    =   {Cambridge (Mass)}, 
  publisher   =   {The MIT Press},
  pagetotal   =   {xviii, 324},
  isbn        =   {0-262-52345-0},
  langid      =   {english}
}
\end{verbatim} 

The \t|edition| field is edition of a printed publication. It's required
if the item is not a first edition. Use integer or literal string to fill in
this field.

The \t|pagetotal| field is total number of pages of the work. If multiple
kinds of numeration is used in the work, e.g. arabic as well as roman numerals,
both can be provided, separated by comma. Only for arabic numerals
the localization term \t|pages| is appended. Note that total number of pages
is no longer required, see also \ref{pkg:opt:iso690:pp}.

The \t|langid| field is required for multilingual support of printing
references. This affects printing of localization terms used in the reference,
e.g. \t|edition| field. See also \ref{gen:multilang}.

\paragraph{Contribution to a collection} \hfill\\

\c{t02}
\begin{verbatim}
@INCOLLECTION{t02,
  author        =   {Greenberg, David}, 
  year          =   {1998},
  title         =   {Camel drivers and gatecrashers},
  subtitle      =   {quality control in the digital research library},
  editor        =   {Hawkins, B.L and Battin, P},
  booktitle     =   {The mirage of continuity},
  booksubtitle  =   {reconfiguring academic information 
                      resources for the 21st  century}, 
  location      =   {Washington (D.C.)}, 
  publisher     =   {Council on Library and Information Resources; 
                      Association of American Universities}, 
  pages         =   {105-116},
  langid        =   {english}
}
\end{verbatim} 

The \t|title| field is title of the contribution, the \t|booktitle|
is title of the collection.

The \t|pages| field is one or more page numbers or page ranges. This field
is essential since the reference should identify the part of the item
that is cited.

It's also possible to use the cross-referencing feature to reference
from a contribution entry (child entry) to a separate entry of collection
(parent entry). This can be done by the field \t|crossref|
as the following example shows:
\newline

\noindent\c{prispevek1}
 
\begin{verbatim}
@COLLECTION{sbornik,
  title       =   {Mimořádně užitečný sborník},
  editor      =   {Geniální, Jiří},
  year        =   {2007},
  langid      =   {czech},
  location    =   {Praha},
  publisher   =   {Academia},
  isbn        =   {978-222-626-222-2}
}

@INCOLLECTION{prispevek1,
  crossref    =   {sbornik},
  title       =   {Velmi zajímavý článek},
  author      =   {Vlaštovka, Josef},
  pages       =   {22-45}
}
\end{verbatim}

Now, there is no need to fill in \t|booktitle| in the \t|sbornik| entry.
The \t|biber| backend program performs the inheritance between parent and
child entry automatically.

\paragraph{Article in a serial}\hfill\\

\c{t03}
\begin{verbatim}
@ARTICLE{t03,
  author        =   {LYNCH, C.},
  year          =   {2005},
  title         =   {Where do we go from here?},
  subtitle      =   {the next decade for digital libraries},
  journaltitle  =   {DLib Magazine},
  volume        =   {11},
  number        =   {7/8},
  urldate       =   {2005-08-15},
  url           =   {http://www.dlib.org/dlib/july05/lynch/07lynch.html}, 
  issn          =   {1082-9873},
  langid        =   {english}
}
\end{verbatim}

The example above is an article in online magazine. If the magazine
is available online only, i.e. it is not published in print,
it should be cited as an electronic information resource. This can be
achieved by providing the \t|urldate| field. On the other hand
\newline

\c{knuth}
\begin{verbatim}
@PERIODICAL{tug,
  journaltitle  =   {TUGBoat},
  publisher     =   {TUG},
  issn          =   {1222-3333},
  langid        =   {english},
  date          =   {1980/},
  url           =   {http://tugboat.tug.org/}
}

@ARTICLE{knuth,
  author        =   {Knuth, Donald},
  title         =   {Journeys of \TeX},
  volume        =   {17},
  number        =   {3},
  year          =   {2003},
  pages         =   {12-22},
  url           =   {http://tugboat.tug.org/kkk.pdf},
  crossref      =   {tug}
}
\end{verbatim}

\noindent the example shows an article in printed magazine, which is also
available online. Similar to the contribution to a collection,
cross-referencing feature is possible to be used.

\subsection{Hints and Caveats}

This section provides additional hints concerning \t|biblatex| package
as well as ISO 690 standard.

For now, some of the things has to be dealt with at the level of \t|bib|
file, another ones are directly addressed in this style package.
Everything else relies on \t|biblatex| package, so please also refer
to the \t|biblatex| documentation.

\subsubsection{Creators}

The persons or organizations responsible for the cited work should be
primarily given into the \t|author| field. If it's not appropriate,
another fields like \t|editor| and \t|editorx| family fields or some
specific ones (e.g. \t|translator|) can be used. Note also the field
\t|editortype| and \t|editortypex| family fields which can be used
to specify type of the editor. This is useful to distinguish the role
of the creator and its relationship to the cited work. Some roles
are supported by default, e.g. \t|editor|, \t|compiler|, \t|founder|,
\t|reviser|, etc., in other cases the literal string can be entered.

Example:
When citing cinematographic works which are typically output of many
individuals, title should be used as the first element of the reference.
However, it's appropriate to include some relevant roles, e.g. director:

%\begin{ltxexample}
\begin{verbatim}
  ..
  editora     = {Welles, Orson}
  editoratype = {Directed by}
  ..
\end{verbatim}
%\end{ltxexample}

in English, or

%\begin{ltxexample}
\begin{verbatim}
  ..
  editora     = {Welles, Orson}
  editoratype = {Réžia}
  ..
\end{verbatim}
%\end{ltxexample}

in Slovak.

The field \t|nameaddon| can be used to append additional information
to the creator's name, e.g. variant forms of a name, name additions,
pseudonyms, etc. This field is printed as is, in square brackets after
creator's name.

For anonymous works cited by author-year method, the term \textit{Anon}
should be used instead of creator's name. Please, reflect this
in \t|bib| file, since there is no other support for this for now.

\subsubsection{Titles}

Similar to the \t|nameaddon| field for names, \t|titleddon| serves such
purpose for titles. This field is appropriate for providing other or
alternative titles, elucidation of ambiguous or incorrect titles,
substitute for no titles, translation of titles, etc.

Note that also other \t|*titleaddon| fields are supported by default.

\subsubsection{Medium type}

The field \t|howpublished| is used for providing information about
medium designation or type of medium. Default value for electronic
information resources is \t|online|. This field is printed as is,
in square brackets after title section, generally.

\subsubsection{Edition}

The \t|edition| field is edition of a publication. It's required
if the item is not a first edition. Use integer or literal string
to fill in this field. Please reflect the constraint, not to print
the edition if the cited item is a first edition, by not providing
this field in the \t|bib| file.

The \t|version| field is used for providing information about
updated versions of an item, usually software.

\subsubsection{Date}

In case no date is given and also no approximation is possible, that
should be stated. Please reflects this in \t|bib| file by including
the following line into the respective entry.

\begin{verbatim}
  ..
  date = {\mkbibbrackets{\bibsstring{nodate}}}
  ..
\end{verbatim}

Explanation of the above code:
\t|nodate| is localization string which prints something like
\textit{n.d.}, \textit{b.r.}, etc. depending on the language,
\t|bibsstring| is a command to use such localization term and
\t|mkbibbrackets| is a command used to wrap it's argument into
the square brackets. Finally, this statement is entered into
the date entry field to be available as a date in a reference.
So as a result we get e.g. \textit{[n.d.]}, which conforms
the standard.

Please also note the syntax for dates -- ISO 8601 format
(YYYY-MM-DD) is accepted. Please use slash instead of
any kind of dash for ranges of dates. If the range is
open ended, enter just the first date followed by a slash.
And last, but not least, use \t|date| field instead of
the \t|year| in general.

Examples follow:

\begin{verbatim}
  date = {2012-12-21}    % exact date
  date = {1998/2001}     % date range
  date = {2016/}         % open ended date range
\end{verbatim}

\nocite{*}
\printbibliography[title={Reference bibliography},heading={bibnumbered}]


\section{Revision history}

\begin{changelog}

\begin{release}{0.2.1}{2016-03-13}
\item Issues about punctuation marks and redundant spaces solved
\item Compatibility support for the latest version the \t|biblatex| package
\item Reformatted the driver for \verb|inbook| entry type
\end{release}

\begin{release}{0.2}{2015-03-25}
\item Gathered changes during past four years
\item Issue about the spacing of strings in main document language solved
\end{release}

\begin{release}{0.1}{2011-02-03}
\item First release
\item Draft of documentation
\item Support for almost all of the entry types provided by the \t|biblatex| package
\end{release}

\end{changelog}

\end{document}
