\documentclass[a4paper,10pt]{ltxdockit}
\usepackage[english,czech]{babel}
\usepackage[utf8]{inputenc}
%\usepackage{tgpagella}
\usepackage[T1]{fontenc}
\usepackage{microtype}
\usepackage{hanging}


\def\t|#1|{\texttt{#1}}
\def\c#1{%
\hangpara{3em}{1}%
\fullcite{#1}}

\usepackage[
   backend=biber
  ,style=iso-authoryear
  ,autolang=other
  ,pagetotal=true
  ,sortlocale=cs_CZ
  ,bibencoding=UTF8
  ,spacecolon=true
  %,block=ragged
]{biblatex}
\addbibresource{mybib.bib}
\titlepage{%
  title={Styl ISO 690 pro biblatex},
  subtitle={},
  url={https://github.com/michal-h21/biblatex-iso690},
  author={Michal Hoftich},
  email={michal.h21@gmail.com},
  revision={0.2.1},
  date={\today}}

\begin{document}
\printtitlepage
\tableofcontents
\section{Úvod}
\t|biblatex| je systém správy bibliografických citací a odkazů pro typografický
systém \LaTeX. Vychází ze staršího systému \t|bibtex| a nahrazuje ho.
\t|biblatex-iso| je styl pro \t|biblatex|, který umožňuje citovat literaturu
podle normy \t|ISO 690|, doporučenou pro použití v českých vysokoškolských
kvalifikačních pracích \parencite{t00}.

\section{Poděkování}

Na vývoji stylu spolupracovali Johaness Bötchner, Moewew, Dávid Lupták a další.

\section{Použití}
\begin{verbatim}
..
\usepackage[english,czech]{babel}
% Nebo, pokud používáte xelatex, místo babelu lze použít 
\usepackage{polyglossia}
\setmainlanguage{czech}
\setotherlanguage{english}
..
\usepackage[
   backend=biber
  ,style=iso-authoryear
  ,sortlocale=cs_CZ
  ,autolang=other
  ,bibencoding=UTF8
]{biblatex}
\addbibresource{menosouborusdatabazi.pripona}
..
\printbibliography

\end{verbatim}

Podle normy se mají některé údaje v bibliografických záznamech uvádět v jazyce dokumentu, jiné v jazyce záznamu. Pokud se odkazujete na dokumenty, psané v různých jazycích, je třeba všechny tyto jazyky nahrát balíčkem \t|babel| nebo \t|polyglossia|. Jazyk také doplňte ke každému záznamu do pole \t|hyphenation|.

%Pro správnou podporu lokalizace je třeba použít balíček babel. V seznamu jazyků se jako poslední uvádí jazyk dokumentu. 
\subsection{Volby}
\begin{description}
\item[style] požadovaný styl. Jsou dostupné tři varianty, \t|iso-authoryear| pro citování tzv. harvardskou metodou, \t|iso-numeric| pro citování pomocí číselných odkazů a \t|iso-authortitle|.
\item[backend] program, kterým se generují bibliografické údaje. \t|biber| je vyvinut speciálně pro \t|biblatex| a má proto největší možnosti. Další možnosti jsou \t|bibtex| a \t|bibtex8|, ale jejich podpora češtiny nedosahuje úrovně \t|biberu|
\item[babel] ovlivňuje vypisování lokalizačních řetězců a dělení slov.
\item[sortlocale] ovlivňuje řazení literatury. Je třeba zadat \t|locale| kód hlavního jazyka dokumentu.
\item[bibencoding] uvádí se, pokud je bibliografická databáze v jiném kódování, než dokument. 

\end{description}


\subsection{Databáze}

\t|biblatex| podporuje množství polí, které ve standartním \t|bibtexu| chyběly. Zde je několik příkladů bibliografických záznamů a jejich \t|bibtexového| kódu.

%\begin{description}

\paragraph{Kniha} \hfill\\

\c{t01}
\begin{verbatim}
@BOOK{t01,
  author      =   {Borgman, Christine L.}, 
  year        =   {2003},
  title       =   {From Gutenberg to the global information infrastructure}, 
  subtitle    =   {access to information in the networked world},
  edition     =   {1},
  location    =   {Cambridge (Mass)}, 
  publisher   =   {The MIT Press},
  pagetotal   =   {xviii, 324},
  isbn        =   {0-262-52345-0},
  hyphenation =   {english}
}
\end{verbatim} 

\t|edition| je vydání, je povinným údajem pouze pokud je jiné, než první. Zapisuje se číslicí nebo slovně. \t|pagetotal| je počet stran. \t|hyphenation| udává jazyk dokumentu, je nutné ho zadat u dokumentů napsaných v jiném, než ve kterém je napsaný hlavní text. Tato volba ovlivňuje výpis některých řetězců, jako je například vydání.

\paragraph{Článek ve sborníku} \hfill\\

\c{t02}
\begin{verbatim}
@INCOLLECTION{t02,
  author        =   {Greenberg, David}, 
  year          =   {1998},
  title         =   {Camel drivers and gatecrashers},
  subtitle      =   {quality control in the digital research library},
  editor        =   {Hawkins, B.L and Battin, P},
  booktitle     =   {The mirage of continuity},
  booksubtitle  =   {reconfiguring academic information 
                      resources for the 21st  century}, 
  location      =   {Washington (D.C.)}, 
  publisher     =   {Council on Library and Information Resources; 
                      Association of American Universities}, 
  pages         =   {105-116},
  hyphenation   =   {english}
}
\end{verbatim} 
 \t|title| je název článku, \t|booktitle| název sborníku. Rozsah stran udávají \t|pages|.
 
 Je možné mít samostatný záznam sborníku a na ten se z článku odkazovat pomocí pole \t|crossref|:
 
\noindent\c{prispevek1}
 
 \begin{verbatim}
@COLLECTION{sbornik,
  title       =   {Mimořádně užitečný sborník},
  editor      =   {Geniální, Jiří},
  year        =   {2007},
  hyphenation =   {czech},
  location    =   {Praha},
  publisher   =   {Academia},
  isbn        =   {978-222-626-222-2}
}

@INCOLLECTION{prispevek1,
  crossref    =   {sbornik},
  title       =   {Velmi zajímavý článek},
  author      =   {Vlaštovka, Josef},
  pages       =   {22-45}
}
 \end{verbatim}
 
 V záznamu \t|sbornik| nemusíme zadávat booktitle, \t|biber| automaticky při vložení do jiného záznamu automaticky převede \t|title| na \t|booktitle|. 
 
 \paragraph{Článek v časopise}\hfill\\
 
 \c{t03}
 \begin{verbatim}
 @ARTICLE{t03,
  author        =   {LYNCH, C.},
  year          =   {2005},
  title         =   {Where do we go from here?},
  subtitle      =   {the next decade for digital libraries},
  journaltitle  =   {DLib Magazine},
  volume        =   {11},
  number        =   {7/8},
  urldate       =   {2005-08-15},
  url           =   {http://www.dlib.org/dlib/july05/lynch/07lynch.html}, 
  issn          =   {1082-9873},
  hyphenation   =   {english} 
}
 \end{verbatim}
 
\noindent Toto je článek v online časopise. Pokud časopis zároveň nevychází i tištěně, cituje se jako online zdroj. Toho docílíme použitím pole \t|urldate|. Naproti tomu 
 
 \c{knuth}
 \begin{verbatim}
 @PERIODICAL{tug,
  journaltitle  =   {TUGBoat},
  publisher     =   {TUG},
  issn          =   {1222-3333},
  hyphenation   =   {english},
  year          =   {1980-},
  url           =   {http://tugboat.tug.org/}
}

@ARTICLE{knuth,
  author        =   {Knuth, Donald},
  title         =   {Journeys of \TeX},
  volume        =   {17},
  number        =   {3},
  year          =   {2003},
  pages         =   {12-22},
  url           =   {http://tugboat.tug.org/kkk.pdf},
  crossref      =   {tug}
}
 \end{verbatim}
 
\noindent zde jde o článek v tištěném časopise, který má i online verzi. Podobně jako u článku ve sborníku, můžeme mít i samostatný záznam pro časopis a jednotlivé články na něj odkazovat.
% \end{description}
%\parencite{t01,t02}
\nocite{*}
\printbibliography[title={Ukázková bibliografie},heading={bibnumbered}]


\section{Historie změn}
\begin{changelog}
\begin{release}{0.2.1}{2016-03-13}
\item Vyřešeny problémy s interpunkcí a nadbytečnými mezerami
\item Řešení kompatibility se současnou verzí Biblatexu
\item Přepracován formát pro \verb|inbook|
\end{release}
\begin{release}{0.2}{2015-03-25}
\item Nakumulované změny za čtyři roky
\item Vyřešen problém s mezerami u tisku řetězců v jazyce dokumentu
\end{release}
\begin{release}{0.1}{2011-02-03}
\item První vydání
\item Pokus o dokumentaci
\item Podpora pro většinu typů záznamů \t|biblatexu|
\end{release}
\end{changelog}
\end{document}
